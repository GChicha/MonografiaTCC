\documentclass[
	12pt,
	oneside,
	a4paper,
	english,
	brazil
]{abntex2}

\usepackage{lmodern}
\usepackage[T1]{fontenc}
\usepackage[utf8]{inputenc}
\usepackage{indentfirst}
\usepackage{color}
\usepackage{graphicx}
\usepackage{microtype}

\usepackage{caption}

\usepackage[brazilian,hyperpageref]{backref}
\usepackage[alf]{abntex2cite}

\usepackage{macros}

\titulo{Previsão de séries temporais por meio de aprendizado de máquina}
\autor{Guilherme Chichanoski}
\local{Maringá}
\data{2018}
\orientador{Valéria Delisandra Feltrim}
\instituicao{Universidade Estadual de Maringá\\
Centro de Tecnologia --- Departamento de Informática\\
Bacharelado em Ciência da Computação}
\tipotrabalho{Trabalho de Conclusão de Curso}

\preambulo{Trabalho de Conclusão de Curso de Graduação apresentado ao 
Departamento de Informática da Universidade Estadual de Maringá, como requisito 
parcial para obtenção do grau de Bacharel em Ciência da Computação.}

\makeatletter
\hypersetup{
		pdftitle={\@title},
		pdfauthor={\@author},
		pdfsubject={\imprimirpreambulo},
		pdfcreator={LaTeX with abnTeX2},
		pdfkeywords={abnt}{latex}{abntex}{abntex2}{projeto de pesquisa},
		colorlinks=true,	% false: boxed links; true: colored links
		linkcolor=blue,		% color of internal links
		citecolor=blue,		% color of links to bibliography
		filecolor=magenta,	% color of file links
		urlcolor=blue,
		bookmarksdepth=4
}
\makeatother

\setlength{\parindent}{1.3cm}
\setlength{\parskip}{0.2cm}

\begin{document}

\frenchspacing

\imprimircapa{}

\imprimirfolhaderosto{}

\textual{}

\pdfbookmark[0]{\contentsname}{toc}
\tableofcontents*
\cleardoublepage

\chapter{Introdução}

Segundo \citeonline{wiley} prever é a tarefa de predizer valores futuros ou 
eventos. Essa tarefa pode não ser trivial considerando que autores e 
organizações famosas já incorreram e ainda incorrem em previsões que se 
demonstram erradas.  Podemos citar como exemplo o The New York Times que previu 
em 1966 que em 2000 existiriam somente 220.000 computadores no Estados Unidos.

Ainda segundo \citeonline{wiley} a tarefa de prever é importante em diferentes 
áreas incluindo governo e industrias. Segundo \citeonline{wiley} as previsões 
são classificas como sendo de curto, médio e longo prazo.  Sendo as de curto 
prazo previsões de curto período, ou seja, somente poucos passos a frente, 
podendo esses passos serem contados em dias, semanas ou meses.  Já previsões de 
médio prazo podem se estender por prazos de até um ano no futuro, enquanto que 
prazos mais longos serão chamados de previsões de longo prazo. Ainda conforme o 
autor, previsões de longo prazo são mais difíceis e suscetíveis a fatores 
externos, entretanto previsões de curto e médio prazo permitem normalmente a 
modelagem utilizando métodos estatísticos, muitas vezes suficientes a tarefa 
empregada.

\postextual

\bibliography{referencias}

\end{document}
