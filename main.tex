\documentclass[
	12pt,
	oneside,
	a4paper,
	english,
	brazil
]{abntex2}

\usepackage{lmodern}
\usepackage[T1]{fontenc}
\usepackage[utf8]{inputenc}
\usepackage{indentfirst}
\usepackage{color}
\usepackage{graphicx}
\usepackage{microtype}

\usepackage{caption}

\usepackage[brazilian,hyperpageref]{backref}
\usepackage[alf]{abntex2cite}

\usepackage{macros}

\titulo{Previsão de séries temporais por meio de aprendizado de máquina}
\autor{Guilherme Chichanoski}
\local{Maringá}
\data{2018}
\orientador{Valéria Delisandra Feltrim}
\instituicao{Universidade Estadual de Maringá\\
Centro de Tecnologia --- Departamento de Informática\\
Bacharelado em Ciência da Computação}
\tipotrabalho{Trabalho de Conclusão de Curso}

\preambulo{Trabalho de Conclusão de Curso de Graduação apresentado ao 
Departamento de Informática da Universidade Estadual de Maringá, como requisito 
parcial para obtenção do grau de Bacharel em Ciência da Computação.}

\makeatletter
\hypersetup{
		pdftitle={\@title},
		pdfauthor={\@author},
		pdfsubject={\imprimirpreambulo},
		pdfcreator={LaTeX with abnTeX2},
		pdfkeywords={abnt}{latex}{abntex}{abntex2}{projeto de pesquisa},
		colorlinks=true,	% false: boxed links; true: colored links
		linkcolor=blue,		% color of internal links
		citecolor=blue,		% color of links to bibliography
		filecolor=magenta,	% color of file links
		urlcolor=blue,
		bookmarksdepth=4
}
\makeatother

\setlength{\parindent}{1.3cm}
\setlength{\parskip}{0.2cm}

\begin{document}

\frenchspacing

\imprimircapa{}

\imprimirfolhaderosto{}

\textual{}

\pdfbookmark[0]{\contentsname}{toc}
\tableofcontents*
\cleardoublepage{}

\chapter{Introdução}

Segundo \citeonline{wiley} prever é a tarefa de predizer valores futuros ou 
eventos. Essa tarefa pode não ser trivial considerando que autores e 
organizações famosas já incorreram e ainda incorrem em previsões que se 
demonstram erradas.  Podemos citar como exemplo o The New York Times que previu 
em 1966 que em 2000 existiriam somente 220.000 computadores no Estados Unidos.

Ainda segundo \citeonline{wiley} a tarefa de prever é importante em diferentes 
áreas incluindo governo e industrias. Segundo \citeonline{wiley} as previsões 
são classificas como sendo de curto, médio e longo prazo.  Sendo as de curto 
prazo previsões de curto período, ou seja, somente poucos passos a frente, 
podendo esses passos serem contados em dias, semanas ou meses.  Já previsões de 
médio prazo podem se estender por prazos de até um ano no futuro, enquanto que 
prazos mais longos serão chamados de previsões de longo prazo. Ainda conforme o 
autor, previsões de longo prazo são mais difíceis e suscetíveis a fatores 
externos, entretanto previsões de curto e médio prazo permitem normalmente a 
modelagem utilizando métodos estatísticos, muitas vezes suficientes a tarefa 
empregada.

Em contrapartida podemos citar técnicas de aprendizado de máquinas que estão se 
tornando cada vez mais notórias pelos bons resultados que obtêm onde são 
aplicadas, no caso de de redes neurais artificiais podemos obter resultados 
superiores aos métodos estatísticos comumente empregado conforme analisado por 
\citeonline{zhang}, o autor ainda apresenta outros exemplos onde a utilização de 
redes neurais obtiveram bons resultados, considerando ainda que o estudo já esta 
datado podemos encontrar em publicações mais resultados ainda melhores.

E em se tratando de previsões uma fonte de informação comumente utilizado são as 
séries temporais, segundo \citeonline{wiley} essas séries são compostas de 
observações sequenciais ao longo do tempo. Por exemplo podemos observar um 
exemplo de série na figura-\ref{serie0}. Sendo essas observações separadas 
unicamente pelo tempo pode-se obter os dados em diferentes intervalos, podendo 
por exemplo ser observações diárias, semanais ou ainda anuais.

A relevância das previsões são percebidas em aplicações onde esta é aplicada 
diretamente no processo de tomada de decisão, podendo auxiliar desde a 
organização de um negócio ao entendimento do comportamento de uma população.

\chapter{Previsão}
Segundo \citeonline{wiley} podemos separar o processo de previsão em diversas 
atividades, colocadas e detalhadas a seguir:

\begin{itemize}
	\item Definição do problema\\
		Envolve definir e entender a tarefa de previsão a ser realizada,
		considerando o prazo a ser previsto e definindo os dados necessários e
		quanto será o intervalo de uma previsão e outra.
	\item Coleta dos dados\\
		Conforme definido na etapa anterior ocorrerá nessa atividade a coleta
		dos dados.
	\item Análise dos dados\\
		Atividade de alta importância para a seleção do modelo mais adequado,
		nessa etapa é utilizada de observações gráficas e extração de dados para
		identificar padrões que ajudem. Ainda ocorrerá a identificação de 
		observações problemáticas e marcadas.
	\item Seleção e verificação do modelo\\
		Consiste da seleção e verificação de como o modelo escolhido se comporta 
		com os dados fornecidos. Para verificar o comportamento será utilizado 
		métricas que permitam comparar o resultado obtidos com outros modelos.
	\item Avaliação do modelo\\
		Etapa para avaliar como o modelo se comportará com novos dados obtidos, 
		normalmente é realizado a partir de dados separados obtidos das mesmas 
		informações utilizada em etapas anteriores, porem uma parte separada 
		somente para este fim.
	\item Publicação do modelo\\
		Com o modelo devidamente selecionado e avaliado o processo é instalado 
		em ambiente de produção, tendo que observar as alterações necessárias 
		para que novos dados sejam inseridos corretamente.
	\item Monitoramento da performance do modelo\\
		Uma ultima atividade que ocorrerá de forma recorrente, deve-se 
		continuamente avaliar como o modelo aplicado se comporta em relação ao 
		ambiente, já que o ambiente é algo volátil.
\end{itemize}

Vale-se notar que após a tarefa de avaliação se essa resultar em um valor 
insatisfatório deve-se retornar a fase anterior e refeita a avaliação até que um 
modelo que obedeça as especificações seja encontrado.

\chapter{Séries temporais}

Como já colocado anteriormente, séries temporais são observações sucessivas ao 
longo do tempo. Vale-se notar que esse tipo de série se caracteriza pelo fato de 
suas observações serem dependentes da observação anterior. Essas séries ainda 
podem demonstrar características como tendência e sazonalidade, sendo que a 
primeira confere a série um comportamento lento que pode levar a observações 
futuras como valores menores ou maiores e a segunda característica confere a ela 
padrão que apresente ciclos, podendo ser estes semanais, mensais, anuais ou um 
período qualquer de tempo.

Uma série é descrita matematicamente pelo conjunto $\{X(t): x \in T\}$, podendo 
ser $t$ um tempo contínuo ou discreto, sendo o primeiro quando se possui as 
observações de todos $t$ possíveis e a segunda quando entre uma observação e 
outra existe um período igual de tempo.

Uma série temporal classicamente é decomposta seguindo a 
equação-\ref{eq:timeseries}, sendo $t$ usado para denotar o tempo, $T$ nos 
fornece a tendência e $C$ a componente sazonal ou cíclica, já $R$ é a componente 
aleatória e normalmente é caracterizada por possuir média zero, denota-se ainda 
como ruído branco.

\begin{equation}
	\label{eq:timeseries}
	X_t = T_t + C_t + R_t
\end{equation}

\subsection{Tendência}

Segundo \citeonline{ehlers} não existe uma definição precisa de tendência, mas 
normalmente é associado ao comportamento de mudança das observações em período 
de longo prazo. Uma série com tendência pode ser descrita com um função também 
sendo a forma apresentada na equação-\ref{eq:tendencia}

\begin{equation}
	\label{eq:tendencia}
	X_t = \alpha + \beta{}t + \epsilon{}_t
\end{equation}

\section{Dados}

\postextual

\bibliography{referencias}

\end{document}
